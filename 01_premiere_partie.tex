% Titre de la premiere partie
\section[Intro]{Introduction et présentation du sujet}

%%%%%%%%%%%%%%%%%%%%%%%%%%%%%%%%%%%%%%%%%%%%%%%%
% Première diapo
%%%%%%%%%%%%%%%%%%%%%%%%%%%%%%%%%%%%%%%%%%%%%%%%
\begin{frame}
	\frametitle{Introduction et présentation du sujet}
	\framesubtitle{Contextualisation}

	\begin{block}{Aujourd'hui}
		Augmentation rapide de la démographie, et problème climatique
	\end{block}

	\pause

	\begin{alertblock}{Problème}
		\pause
		Comment s'assurer que nous utilisons l'espace urbain au maximum de ses capacités ?
	\end{alertblock}

\end{frame}


%%%%%%%%%%%%%%%%%%%%%%%%%%%%%%%%%%%%%%%%%%%%%%%%
% Deuxième diapo
%%%%%%%%%%%%%%%%%%%%%%%%%%%%%%%%%%%%%%%%%%%%%%%%
\begin{frame}
	\frametitle{Introduction et présentation du sujet}
	\framesubtitle{Proposition de modélisation}

	\begin{columns}
		\column{0.5\linewidth} % première colonne
			Afin de modéliser une ville, il faut alors garder les éléments suivants en tête

			\pause

			\begin{itemize}[<+->]
				\item		Les besoins des habitants
				\item		La disponibilité de l'espace
				\item		L'optimisation économique et temporelle

			\end{itemize}



		\column{0.5\linewidth} % 2e colonne
			\begin{alertblock}{Mais la modélisation ne suffit pas}<+->
				Car on veut arriver à une ville \textit{optimisée}!
			\end{alertblock}

	\end{columns}

\end{frame}

\begin{frame}
    \frametitle{Introduction et présentation du sujet}
    \framesubtitle{Proposition d'optimisation}
    \begin{block}{Pour la modélisation}
    \pause
    On utilisera donc le module \texttt{pygame} de \texttt{python} ainsi que des classes pour modéliser les batiments, les habitants et la ville elle-même.
    \end{block}
    \pause
    
    \begin{block}{Et pour l'optimisation}
    \pause
    
    On utilisera le principe d'\textit{apprentissage par renforcement}, pour faire évoluer la ville. 
    \end{block}
\end{frame}


